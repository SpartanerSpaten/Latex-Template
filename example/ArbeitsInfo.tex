
\documentclass{article}
    % General document formatting
    \usepackage[margin=0.7in]{geometry}
    \usepackage[parfill]{parskip}
    \usepackage[utf8]{inputenc}
    \usepackage{hyperref}
    
    % Related to math
    \usepackage{amsmath,amssymb,amsfonts,amsthm}
    \usepackage{scrpage2}
	\pagestyle{scrheadings}
	\usepackage{graphicx}
		
	% Kopfzeile
	\clearscrheadfoot
	\ihead{Tassilo Tanneberger}
	\chead{SRZ Aktueller Arbeits Stand}
	\ohead{10.5.2020}
	
	\ofoot{\pagemark}
	
	\newcommand{\RN}[1]{%
  	\textup{\uppercase\expandafter{\romannumeral#1}}%
	}

	
	\graphicspath{ {./resources/} }


\begin{document}


\section*{ Vorstellung des Projektes }
Ich würde hier erst mal auf die mein Vorstellungdokument verweisen da sich fundamentale Sachen seitdem nicht verändert haben.


\section*{ Akueller Entwicklungs Stand }


Der aktuelle Entwicklung Stand ist immer unter \url{https://bitbucket.org/revol-xut/workspace/projects/RUSSEL} einsehbar wobei alle 5 dort gelisteten Repositories zum Projekt gehören:

\begin{itemize}
	\item Facharbeit LaTeX repo für Theoretischen Teil
	\item Russel-Interpreter Byte Code Interpreter als Static Linked Library 
	\item Russel Eigentlicher Unix Daemon und Haupt Repo
	\item Russel-CLI Russel Command Line Tool
	\item Russel-Python-Interface API wrapper für einfache Integration sowie Benchmarks in den Examples	
\end{itemize}

Funktional ist der aktuelle Entwicklungsstand so, dass der einzelne Daemon Aufgaben lösen kann und auch zurückschicken. Es findet aktuell bloß keine Kommunikation zwischen Daemons statt.

\begin{figure}[h]
	\centering
			
	 \includegraphics[scale=0.8]{Result_Speed_Test}
\end{figure}

Das ist ein Beispiel Run des Speedtestskriptes\footnote{\url{https://bitbucket.org/revol-xut/russel-python-interface/src/master/examples/speed_test.py}} was ich aktuell für experimentelle Benchmarkings und Tests nutze. Das kleine Script sendet eine quadratische Matrix und multipliziert sie mit einem gegeben skalar.


\section*{ Zeitplan einordnung }

Meine Todo liste für die nächsten Wochen sieht so aus:

\begin{enumerate}
	\item .[BUG] occasional message loss
	\item .[FEATURE] Thread for recv.
	\item .[FEATURE] Daemon representaion in Scheduling Sys.
	\item .[FEATURE] Work Balancer
	\item .[OPTIMIZATION] memcpy entfernen und dealloc umorganisieren.
	\item .[FEATURE] Daemon Status erfahren und darauf reagieren.
	\item .[FEATURE] Group Tasks und Variable Caching	
\end{enumerate}


Für meine Todo liste würde ich mit Schulbeginn ungefähr 2-3 Wochen einplanen. Und für das Gesamten Praktischen Teil ungefähr 1 Monat. Somit würde ich bezweifeln das die Arbeit in einem zufriedenstellenden Zustand bis zum 20.5.20 fertig wird

\end{document}
