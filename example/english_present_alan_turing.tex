\documentclass[10pt,]{beamer}
\usepackage[utf8]{inputenc}
\usepackage{wrapfig}

\usetheme{Dresden}
\usefonttheme[onlylarge]{structurebold}
\setbeamerfont*{frametitle}{size=\normalsize,series=\bfseries}
\setbeamertemplate{navigation symbols}{}

\title{Why the UK could be considered the birthplace of computer science}
\author{Tassilo Tanneberger}
\date{\today}

\graphicspath{{./resources/}}


\begin{document}
\maketitle

\section{Blectchley Park (G.C. and C.S.) and WW2}
\begin{frame} %%Eine Folie
	
  	\frametitle{Bletchley Park (G.C. and C.S.) and WW2 } %%Folientitel
	
	\begin{figure}		 
    	\includegraphics[width=0.3\linewidth]{bletchley_park_map.jpg}
    	\caption{Geographical location of bletchley park}
    	\label{fig:blectchley_park}
	\end{figure}		
	
	\begin{itemize}
		 \item Center of Allied code-breaking (decryption) during the Second World War. 
		 
		 \item Penetrated Encryption of the Axis powers this includes the lorenz cipher, Siemens and Halske T52 and Enigma.
		 
	\end{itemize}
	
\end{frame}

\section{Workload}
\begin{frame}[fragile] %%Eine Folie
  	\frametitle{Workload} %%Folientitel

	\begin{figure}		 
    	\includegraphics[width=0.6\linewidth]{work_load.png}
    	\caption{Average daily number of Signals to Commands Abroad }
		\label{fig:solve_statistics}
	\end{figure}
	
	Suplied the military with essential information e.g. about submarines.
	
\end{frame}


\section{ Achievements of Bletchley Park }
\begin{frame}[fragile] %%Eine Folie
  	\frametitle{Achievements of Blectchley Park} %%Folientitel
  	
  	\begin{itemize}
  		\item	Colossus first electronic digital programmable computer (before ENIAC) but not turing-complete built by Tommy Flowers. Was used to break the lorenz cipher.
  		\item Big advancements in cryptanalysis e.g. Known-Plaintext-Attacks (see cribs)
		\item 	\begin{equation}
				\frac{5!}{(5-3)!} * 26^3 * \frac{26!}{(26-20)! * 2 ^ {10} * 10! } = 158.962.555.217.826.360.000
				\end{equation}
				All possible ways the enigma could be configured.
  	\end{itemize}

\end{frame}

\section{ Alan Turing }
\begin{frame}[fragile] %%Eine Folie
  	\frametitle{Alan Turing} %%Folientitel
	\begin{wrapfigure}{r}{}
    	\includegraphics[width=0.3\linewidth]{Alan_Turing.jpg}
    	\caption{Alan Turing-1938}
    	\label{fig:alan_turing}
	\end{wrapfigure}		
	Published the paper "Computable Numbers, with an Application to the Entscheidungsproblem“ (28. Mai 1936) where he described the turing maschine and proofed that every problem that can be solved by a algorithm can be solved by the turing maschine. See computability theory. Furthermore did he show (proofe) that the Hilbert Entscheidungsproblem(10th) can have no solution.	\newline \newline
	Because of his homosexuality he forcefully got sterilized by the UK Government. Suffered from depression the following years and committed suicide in 1954 (Age: 41).
  	
\end{frame}

\section{ Questions }
\begin{frame}[fragile] %%Eine Folie
\frametitle{Questions} %%Folientitel
	\begin{itemize}
		\item What cipher did the colossus decrypted ?
		
		\item What are areas alan turing conducted research in ?
	\end{itemize}		
\end{frame}


\section{ Resources}
\begin{frame}[fragile] %%Eine Folie
\frametitle{Resources} %%Folientitel
 Figure \ref{fig:blectchley_park} Geographical Location of Bletchley Park Source: \url{https://upload.wikimedia.org/wikipedia/commons/thumb/0/0f/Buckinghamshire_UK_relief_location_map.jpg/220px-Buckinghamshire_UK_relief_location_map.jpg} [1.3.2020]. \newline
 
 Figure \ref{fig:solve_statistics} Statistic about daily workload of bletchley park \url{https://upload.wikimedia.org/wikipedia/commons/thumb/0/09/Ultra_Hut3_Graph.png/330px-Ultra_Hut3_Graph.png} [1.3.2020].  \newline
 
 Figure \ref{fig:alan_turing} Image of Alan Turing when he worked in bletchley park \url{https://upload.wikimedia.org/wikipedia/commons/thumb/7/79/Alan_Turing_az_1930-as_%C3%A9vekben.jpg/220px-Alan_Turing_az_1930-as_%C3%A9vekben.jpg} [1.3.2020].
\end{frame}


\section{ Refrences}
\begin{frame}[fragile] %%Eine Folie
	\frametitle{Refrences} %%Folientitel
		
	\begin{thebibliography}{9}
\bibitem{turing} 
Turing, A.M. (1936), "On Computable Numbers, with an Application to the Entscheidungsproblem", Proceedings of the London Mathematical Society, 2 (published 1937)

\bibitem{wiki_1} 
Wikipedia Article on Alan Turing \\\texttt{\url{https://en.wikipedia.org/wiki/Alan_Turing}} [1.3.2020]

\bibitem{wiki_2} 
Wikipedia Article on Blectchley Park \\\texttt{\url{https://en.wikipedia.org/wiki/Bletchley_Park}} [1.3.2020]

\bibitem{blectchley_park_website} 
Offical Blectchley Park Website \\\texttt{\url{https://bletchleypark.org.uk/our-story}} [1.3.2020]

\end{thebibliography}	
\end{frame}


\end{document}