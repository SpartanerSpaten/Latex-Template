\documentclass[english,ttfont=true]{tudscrmanual}
\usepackage[utf8]{inputenc}

%==========================================================

\usepackage{amsmath,amssymb,amsfonts,amsthm}
\usepackage{scrlayer-scrpage}
\usepackage{tikz}
\usetikzlibrary{arrows.meta}
\usepackage{algorithm}
\usepackage{algpseudocode}
\usepackage{listings}
\usepackage{xcolor}
\usepackage{color}
\usepackage{textcomp}
\usepackage{siunitx}
\usepackage{hyperref}

\setkomafont{section}{\huge\bfseries}

%==========================================================

\pagestyle{scrheadings}

\title{Transforming Matchstickgraphs}
\author{Tassilo Tanneberger}

\newcommand{\linia}{\rule{\linewidth}{1pt}}
\makeatletter
\renewcommand{\maketitle}{\begin{center}
\Huge \@title\end{center}
\linia\\
{\large\@author\hfill\@date\\}}

%==========================================================

\definecolor{codegreen}{rgb}{0,0.6,0}
\definecolor{codegray}{rgb}{0.5,0.5,0.5}
\definecolor{codepurple}{rgb}{0.58,0,0.82}
\definecolor{backcolour}{rgb}{0.95,0.95,0.92}

\definecolor{dkgreen}{rgb}{0,0.6,0}
\definecolor{gray}{rgb}{0.5,0.5,0.5}
\definecolor{mauve}{rgb}{0.58,0,0.82}

\lstset{frame=tb,
  aboveskip=3mm,
  belowskip=3mm,
  columns=flexible,
  basicstyle={\small\ttfamily},
  numbers=none,
  numberstyle=\tiny\color{gray},
  keywordstyle=\color{blue},
  commentstyle=\color{dkgreen},
  stringstyle=\color{mauve},
  breaklines=true,
  tabsize=3,
}


%==========================================================

\begin{document}

\maketitle

\noindent Given is a matchstick graph $ G = \langle V, E \rangle $ where every angle of a edge to the reference plane is a multiple of $\ang{30} $. Read in the graph $ G $ in a data structure defined by you and transform $ G $ so that it becomes a cyclic matchstick graph. That means the every vertex must have at least two attached edges. The resulting graph called $ G_T $ should not contain any unconnected subgraphs. The newly generated graph $ G_T $ should be created by moving the least amount of edges. Display $ G_T $ afterwards.

\subparagraph*{1.)} A minimal amount of potential $ G_T $ candidates should be generated.

\subparagraph*{2.)} Keep the run-time as low as possible. \newline

\noindent This task is heavily modified version of task 4. from the 39. BWINF competition ( see here for more \href{https://bwinf.de/fileadmin/bundeswettbewerb/39/Bundeswettbewerb-Aufgabenblatt.pdf}{information} ).

\subsection*{Example}

\begin{figure}[h]
\caption{exemplary matchstick graph with 4 edges}
\center
\begin{tikzpicture}[auto,node distance=1.5cm]
    	\tikzstyle{main node}=[circle,fill=gray!15,draw, font=\sffamily\Large\bfseries,minimum size=0mm]
    	\tikzstyle{--}=[draw=black!, line width=0.1cm]
  	\node[main node] (a) {};
  	\node[main node] (b) [ below of = a] {};
  	\node[main node] (c) [ below right of = b] {};
  	\node[main node] (d) [ below left of = b] {};
  	
  	\draw(a)--(b);
 	\draw(b)--(c);
 	\draw(b)--(d); 
 	\draw(d)--(c); 
	
	\draw[-{Triangle[width=18pt,length=8pt]}, line width=10pt, color=red](3,-1.3) -- (5, -1.3);
	\node[main node] at (7, -0.5) (a2) {};
  	\node[main node] (b2) [ below of = a2] {};
  	\node[main node] (c2) [ right of = a2] {};
  	\node[main node] (d2) [ below of = c2] {};
  	
  	\draw(a2)--(b2);
 	\draw(a2)--(c2);
 	\draw(b2)--(d2); 
 	\draw(c2)--(d2); 
\end{tikzpicture}

\end{figure}
\end{document}

