\documentclass[12pt]{article}
\usepackage[utf8]{inputenc}
\usepackage[margin=.50in]{geometry}
\usepackage{textgreek}

    % Related to math
\usepackage{amsmath,amssymb,amsfonts,amsthm}
\usepackage{scrlayer-scrpage}
\usepackage{multicol}

\usepackage{tikz}

\pagestyle{scrheadings}

\title{Infinitesimalrechnung}
\author{Tassilo Tanneberger}
\newcommand{\linia}{\rule{\linewidth}{0.5pt}}
\makeatletter
\renewcommand{\maketitle}{\begin{center}
\huge \@title\end{center}
\linia\\
{\large\@author\hfill\@date\\}}

\begin{document}

\maketitle

\section{Differential Rechnung}

\subsection{ Differenzenquotient und Differentialquotient}

Ist \( f: D_f \rightarrow \mathbb{R} \) wobei \( D_f \subset \mathbb{R} \) und \( [x_0; x_1] \subset D_f \) so nennt man den Quotienten Differenzenquotient von \( f \) im Intervall \( [x_0; x_1] \)

\begin{align*}
	\varphi (x_1, x_0) = \frac{f(x_1) - f(x_0)}{x_1 - x_0} = \frac{\Delta y}{\Delta x} \textit{ oder } \ \varphi(x, h) = \frac{f(x + h) - f(x)}{h}
\end{align*}

Der Differentialquotient ist definiert als:

\begin{align*}
	f' (x_0) = \lim_{h \rightarrow 0} \frac{f(x_0 + h) - f(x_0) }{h} = \lim_{x_1 \rightarrow x_0} \frac{f(x_1) - f(x_0)}{x_1 - x_0}
\end{align*}

\begin{figure}[h]


\begin{tabular}{ |p{2cm}| p{10cm}| }
 \hline
 \multicolumn{2}{|c|}{Differenzenquotient} \\
 \hline
\( f(x) \) & Differenzenquotient \\
 \hline
  \( c \) & 0\\
  \( a \cdot x \) & \( a \)  \\
  \( x^2 \) & \( x_0 + x_1 \)  \\
  \( x^3\) & \( x_1^2 + x_1 \cdot x_0 + x_0^2 \)) \\
  \( x^n \) & \( \sum_{ i = 0} ^ {n - 1} x_1^i \cdot x_0^{n - 1 -i} \) \\
  \( exp(x) \) & \( exp(x_0) \cdot \frac{exp(x_1 - x_0) - 1}{x_1 - x_0} \)\\

 \hline
\end{tabular} 
\centering
\end{figure}

Weiterhin sollte der Vorwärtsdifferenzenquotient (siehe oben) , Rückwärtsdifferenzenquotient und Zentraler Differenzenquotient genannt werden.


\subsection{ Differenzierbarkeit }

Die Funktion \( f \) ist an der Stelle \( x \) differenzierbar wenn der beidseitige Grenzwert des Differentialquotienten gleich ist.
	\begin{align*}
		\lim_{x_1 \rightarrow x_0}  \frac{f(x_1) - f(x_0)}{x_1 - x_0} = \lim_{h \rightarrow 0} \frac{f(x_0 + h) - f(x_0)}{h}
	\end{align*}

\newpage

\subsection{ Differenziationsregeln}

\begin{figure}[h]
\def\arraystretch{2}\tabcolsep=3pt
\begin{tabular}{ |p{2cm}| p{10cm}|  }
 \hline
 \multicolumn{2}{|c|}{Differenziationsregeln} \\
 \hline
\( f(x) \) & \( f'(x) \) Ableitung\\
 \hline
  \( a \cdot x^n \)  & \( a \cdot n \cdot x^{n - 1} \) \\
  \( u(x) + v(x) \) &  \( v'(x) + u'(x) \) \\
  \( u(x)  \cdot v(x) \) & \( u'(x) \cdot v(x) + u(x) \cdot v'(x) \) \\
  \( \frac{u(x)}{v(x)} \) & \( \frac{u'(x) \cdot v(x) - u(x) \cdot v'(x)}{v(x) ^ 2} \) \\
  \( u(v(x)) \) & \( v'(x) \cdot u'(v(x)) \) \\
  \( e^x \) & \( e^x \) \\
  \( a^x \) & \( a^x \cdot \ln a \) \\
 \( \ln x \) & \( \frac{ 1 }{ x } \) \\
 \( \sin x \) & \( \cos x \) \\
 \( \cos x \) & \( - \sin x \) \\

 \hline
\end{tabular} 
\centering
\end{figure}

\subsection{ Tangenten }

Tangente an der Funktion \( f(x) \) Stelle \( x \)

\begin{enumerate}
	\item \( t(x) =  f'(x) \cdot x + f(x) \)
	\item Von einem Punkt \( P(0 \vert y)  \) aus: \( f(x) = f'(x) \cdot x + P_y\)
	\item Von einem Punkt \( P(x \vert y) \) aus: \( \frac{f(x) - P_y}{x - P_x} = f'(x) \)
\end{enumerate}

\subsection{ Extrempunkte }

\paragraph{ Extrempunkte } \( f'(x_E) = 0 \land f''(x_E) \neq 0  \)  
\begin{align*}
	\bigg \lbrace \begin{matrix}
		\textit{max } & f''(x) < 0 \\
		\textit{min } & f''(x) > 0
	\end{matrix}
\end{align*}

Wenn \( f''(x_E) = 0 \) ist es kein Extrempunkt sondern ein Sattelpunkt.

\paragraph{Wendepunkte} Sind die Extrempunkte der ersten Ableitung.
\section{Integral Rechnung}

\subsection{Definition des Integrals}

$ \mathcal{Z} = \lbrace a = x_0, x1, \dots, x_n = b \rbrace$

\begin{multicols}{2}
\begin{tikzpicture}
  \draw[thin,gray!40] (-1,-1) grid (6,5);
  \draw[<->] (-1,0)--(6,0) node[right]{$x$};
  \draw[<->] (0,-1)--(0,5) node[above]{$y$};
  
  \draw[scale=0.5, domain=-1:7, smooth, variable=\x, blue] plot ({\x}, {(-0.5*((\x-3)*(\x-3))) + 8});  
  
  \filldraw[fill=green!20, opacity=0.5] (-0.5,0) node[anchor=north]{}
  -- (0,0) node[anchor=north]{}
  -- (0,1.75) node[anchor=north]{}
  -- (-0.5,1.75) node[anchor=south]{}
  -- cycle;
  \filldraw[fill=green!20, opacity=0.5] (0,0) node[anchor=north]{}
  -- (0.5,0) node[anchor=north]{}
  -- (0.5,3) node[anchor=north]{}
  -- (0,3) node[anchor=south]{}
  -- cycle;
  \filldraw[fill=green!20, opacity=0.5] (0.5,0) node[anchor=north]{}
  -- (1,0) node[anchor=north]{}
  -- (1,3.75) node[anchor=north]{}
  -- (0.5,3.75) node[anchor=south]{}
  -- cycle;
  \filldraw[fill=green!20, opacity=0.5] (1,0) node[anchor=north]{}
  -- (1.5,0) node[anchor=north]{}
  -- (1.5,4) node[anchor=north]{}
  -- (1,4) node[anchor=south]{}
  -- cycle;
  \filldraw[fill=green!20, opacity=0.5] (1.5,0) node[anchor=north]{}
  -- (2,0) node[anchor=north]{}
  -- (2,4) node[anchor=north]{}
  -- (1.5,4) node[anchor=south]{}
  -- cycle;
  \filldraw[fill=green!20, opacity=0.5] (2,0) node[anchor=north]{}
  -- (2.5,0) node[anchor=north]{}
  -- (2.5,3.75) node[anchor=north]{}
  -- (2,3.75) node[anchor=south]{}
  -- cycle;
  \filldraw[fill=green!20, opacity=0.5] (2.5,0) node[anchor=north]{}
  -- (3,0) node[anchor=north]{}
  -- (3,3) node[anchor=north]{}
  -- (2.5,3) node[anchor=south]{}
  -- cycle;
  \filldraw[fill=green!20, opacity=0.5] (3,0) node[anchor=north]{}
  -- (3.5,0) node[anchor=north]{}
  -- (3.5,1.75) node[anchor=north]{}
  -- (3,1.75) node[anchor=south]{}
  -- cycle;
\end{tikzpicture}


$\overline{S}(f, \mathcal{Z}) =  \sum_{k=1}^{n} M_k(x_k - x_{k-1})$


\columnbreak
\begin{tikzpicture}
  \draw[thin,gray!40] (-1,-1) grid (6,5);
  \draw[<->] (-1,0)--(6,0) node[right]{$x$};
  \draw[<->] (0,-1)--(0,5) node[above]{$y$};
  
  \draw[scale=0.5, domain=-1:7, smooth, variable=\x, blue] plot ({\x}, {(-0.5*((\x-3)*(\x-3))) + 8});  
  
  \filldraw[fill=red!20, opacity=0.5] (0,0) node[anchor=north]{}
  -- (0.5,0) node[anchor=north]{}
  -- (0.5,1.75) node[anchor=north]{}
  -- (0,1.75) node[anchor=south]{}
  -- cycle;
  \filldraw[fill=red!20, opacity=0.5] (0.5,0) node[anchor=north]{}
  -- (1,0) node[anchor=north]{}
  -- (1,3) node[anchor=north]{}
  -- (0.5,3) node[anchor=south]{}
  -- cycle;
  \filldraw[fill=red!20, opacity=0.5] (1,0) node[anchor=north]{}
  -- (1.5,0) node[anchor=north]{}
  -- (1.5,3.75) node[anchor=north]{}
  -- (1,3.75) node[anchor=south]{}
  -- cycle;
  \filldraw[fill=red!20, opacity=0.5] (1.5,0) node[anchor=north]{}
  -- (2,0) node[anchor=north]{}
  -- (2,3.75) node[anchor=north]{}
  -- (1.5,3.75) node[anchor=south]{}
  -- cycle;
  \filldraw[fill=red!20, opacity=0.5] (2,0) node[anchor=north]{}
  -- (2.5,0) node[anchor=north]{}
  -- (2.5,3) node[anchor=north]{}
  -- (2,3) node[anchor=south]{}
  -- cycle;
  \filldraw[fill=red!20, opacity=0.5] (2.5,0) node[anchor=north]{}
  -- (3,0) node[anchor=north]{}
  -- (3,1.75) node[anchor=north]{}
  -- (2.5,1.75) node[anchor=south]{}
  -- cycle;
\end{tikzpicture}

$\underline{S}(f, \mathcal{Z}) =  \sum_{k=1}^{n} m_k(x_k - x_{k-1})$

\end{multicols}

Umkehroperation zur Differenzierung.

\paragraph{Grundintegrale}

\begin{align*}
	&\int a \cdot x^b \,dx = \frac{a}{b+1} \cdot x^{b+1}  + c \\
	&\int \frac{1}{x} \,dx = \ln \vert x \vert + c\\
	&\int \sin(x) \,dx = \cos(x) + c\\
	&\int \cos(x) \,dx = -1 \cdot \sin(x) + c\\
	&\int a^x \,dx = \frac{1}{\ln a} \cdot a^x + c\\
	&\int \ln x \,dx = x \cdot \ln x - x + c\\
\end{align*}

\end{document}

