\documentclass[12pt]{article}
\usepackage[utf8]{inputenc}
\usepackage[margin=0.8in]{geometry}

    % Related to math
\usepackage{amsmath,amssymb,amsfonts,amsthm}
%\usepackage{scrlayer-scrlayer}
\usepackage{scrlayer-scrpage}
\usepackage{tikz}
\usepackage{multicol}
\usepackage{algorithm}
\usepackage{algpseudocode}

\usepackage{enumerate}
\pagestyle{scrheadings}

\title{Geschichte}
\author{Tassilo Tanneberger}
\newcommand{\linia}{\rule{\linewidth}{1pt}}
\makeatletter
\renewcommand{\maketitle}{
\begin{center}
\huge \@title
\end{center}
\linia\\
{\large\@author\hfill 3.11.2020\\}}

\begin{document}

\maketitle

\subsection*{Wichtige Daten}


\begin{itemize}
	\item 8. Mai 1945 Stunde Null 
	\item 5. Juni 1945 Alliierten Kontrollrat 
	\item 17. Juli - 2. August 1945 Potsdamer Konferenz
	\item 20. Nov 1945 Nürnbergerprozess 
\end{itemize}

\subsection*{Stunde Null}

\paragraph*{Alliierten Kontrollrat}
Unterschiedliche Interessen:
\begin{itemize}
	\item USA 
	\begin{itemize}
		\item Demokratische Umerziehung
		\item Zerschlagen Deutschlands als Militärmacht-
	\end{itemize}
	
	\item GB
	\begin{itemize}
		\item Deutschland als Bollwerk gegen den Kommunismus
		\item Verhindern eines Deutschen Kommunismus
		\item Deutsche Westintegration
	\end{itemize}
	
	\item FR
	\begin{itemize}
		\item Sicherheits und Revenge Bedürfnis $ \Rightarrow $ Nachhaltige Schwächung Deutschlands 
		\item Ablehnen eines Zentral Regierten Deutschlands
		\item Saarland Anextieren und Ruhrgebiet internationalisieren
	\end{itemize}
	
	\item UDSSR
	\begin{itemize}
		\item Schwächung Deutschlands
		\item Reperationen
		\item politische Beeinflussung sowie Kontrolle 
	\end{itemize}
\end{itemize}

\paragraph*{Potsdamer Konferenz} Beschlüsse 

\begin{itemize}
	\item Denazifizierung
	\item Demilitarisierung
	\item Dezentralisierung
	\item Demoralisierung
	\item Demontage
\end{itemize}

\subsection*{Begin des Kaltenkrieges}

Ziele der USA und UDSSR 1945.
\begin{multicols}{2}
UDSSR
\begin{itemize}
	\item Ruhrgebiet von allen Überwacht
	\item Wirtschaftsaufbau für Reperationen
	\item Gesamt Deutscher Staat
	\item Eigenständigkeit
	\item Entmilitarisierung $ \Rightarrow $ Friedliebend
\end{itemize}

\columnbreak

USA
\begin{itemize}
	\item Demokratisch
	\item Einheitliches Wirtschaftsgebiet
	\item Entmilitarisierung
	\item potenzieller Wirtschaftspartner
	\item Eigenständigkeit
	\item Zivilgesellschaft
	\item kein Spielball der Weltpolitik

\end{itemize}

\end{multicols}

Begin des Kaltenkrieges und des finanziellen und militärischen Enagement der USA in der Containment Politik.

\paragraph*{Truman-Doktrin}
\verb|"| freien Völkern beizustehen, die sich der angestrebten Unterwerfung durch bewaffnete Minderheiten oder durch äußeren Druck widersetzen \verb|"| $\Rightarrow$ Ziel der Doktrin war es, die Expansion der Sowjetunion aufzuhalten, und Regierungen im Widerstand gegen den Kommunismus zu unterstützen. 

\paragraph*{Marshall-Plan o. European Recovery Program ERP}

Ziele:
\begin{itemize}
	\item Versorgung Europas
	\item Containment Politik Einflüsse der Sowjetunion durch Wirschaftlichen wohlstand verhindern.
	\item Absatzmärkte
\end{itemize}

\subsection*{Gründung der DDR}

Regierung: SMAD (Sowjetische militärische administration Deutschlands)

\begin{itemize}
	\item 10.6.1945 Bildung Antifaschistischer Demokratischer Block durch SMAD (Willhelm Piek, Walter Ulbricht)
	\item 22.4.1946 Zwangsfusionierung von SPD + KPD $\Rightarrow$ SED
	\item Verhinderung bürgerlicher Demokratie
	\item 7. Okt 1949 Zweiter Deutscher Volksrat gründet DDR inklusive Provisorischer Volkskammer
	\item 30. Mai 1950 Inkraft treten der Verfassung
	\item Wahlen mit Einheitslisten
	\item Macht Übergabe SMAD an Willhelm Pieck
\end{itemize}

Volkskongress: West und Ost Kommunisten
\begin{itemize}
	\item Zentraler Deutscher Staat
	\item 2. Volkskongress Auftrag erarbeiten einer Verfassung
	\item 3. Volkskongress Abnicken der Verfassung
\end{itemize}

Antifaschistische Demokratische Umwälzung (1949 - 1952)
\begin{itemize}
	\item  Antifaschismus
	\item Freundschaft mit UDSSR
	\item Heroisierung der Arbeit
	\item Führung des Proletariats
	\item 5-Jahresplan
	\item Staatseigene Betriebe 
	\item Gleichschaltung nicht komunistischer Partein
\end{itemize}

1952 Nachablehnen der Stalin-Notes: Planmäßiger Aufbau des Sozialismus und Ost-Integration

SED Partei neuen Typs
\begin{itemize}
	\item Nicht aus Wahlen sondern durch Lenins Parteintheorie
	\item Bewusste Vorhut der Arbeiterklasse
	\item Sozialistische Intellektuelle 
	\item demokratischen Zentralismus
	\item Wahl der übergeordnete Parteiorgane durch die Mitglieder
	\item Staffer Parteidisziplin 
	\item Einbringen der Ideologie in den Volkskörper
	\item Totalitätsanspruch des Zentrallkomitess der SED
\end{itemize}

\subsection*{Gründung der BRD}

\begin{itemize}
	\item 27.8.1945 Bildung von Partein
	\item Juni 1947 Marshall-Plan $\Rightarrow$ 3.4.1948 Trifft in Kraft
	\item 23. Feb. 1948 6 Mächtekonferenz in London Frankfurter Dokumente
	\begin{itemize}
		\item Förderales System
		\item Rechtsstaatlich
	\end{itemize}
	\item 20.3.1948 letzts treffen des Allierten Kontrollrates
	\item 20.6.1948 Währungsreform
	\item 14.8.1949 Erste Bundestagswahl Konrad Adenauer
	\item hohe Komissare
	\item Ziele Adenauers
	\begin{enumerate}
		\item West Integration
		\item Sicherheit und pol. Gleichberechtigung
		\item Völkerrechtliche Sourveränität
		\item Deutsche Einheit
	\end{enumerate}
	\item Korea Krieg wieder bewaffnung Deutschlands
	\item Ablehnen Stalin Noten
\end{itemize}

\subsection*{Stalin-Notes}

\begin{itemize}
	\item Deutscher Staat
	\item Besatzung Abziehen 
	\item Grundrechte
	\item Partei Freiheit
	\item Bekämpfen undemokratischer Elemente
	\item k. Militärbündnis Neutralität
	\item Grenzen wie auf der Potsdamer Konferenz festgelegt
	\item Nur kleine Armee
\end{itemize}

\end{document}
