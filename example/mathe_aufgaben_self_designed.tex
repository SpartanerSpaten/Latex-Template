\documentclass{article}
    % General document formatting
    \usepackage[margin=0.7in]{geometry}
    \usepackage[parfill]{parskip}
    \usepackage[utf8]{inputenc}
    
    % Related to math
    \usepackage{amsmath,amssymb,amsfonts,amsthm}
    \usepackage{scrpage2}
	\pagestyle{scrheadings}
	
	% Kopfzeile
	\clearscrheadfoot
	\ihead{Tassilo Tanneberger}
	\chead{Mathe Referenz}
	\ohead{20.3.2020}
	
	\ofoot{\pagemark}
	
	\newcommand{\RN}[1]{%
  	\textup{\uppercase\expandafter{\romannumeral#1}}%
	}


\begin{document}

\subsubsection*{Aufgabe1}

Zeige das du aus zwei koplanaren Geraden \( g_1 \) und \(g_2\) Die Gerade \( f \) berechnen kannst der auf \( g_1  \land g_2 \) senkrecht steht ohne der verwendung des Vektorprodukts.

\( f \perp g_1 \land f \perp g_2 \) 


\subsubsection*{Aufgabe2}
Wie ist es Mathematisch zu interpretieren wenn das Vektorprodukt zweier Vekoren der Nullvektor ist.

\subsubsection*{Aufgabe3}

Gegeben ist die Geradeschar \( g_j \dots \overrightarrow{x} = \begin{pmatrix}
	j \\\ 1 \\\ 1
\end{pmatrix} \cdot r + \begin{pmatrix}
4 \\\ 3 \\\ 1
\end{pmatrix} \) Berechnen sie alle Schnittpunkte mit der Ebene \newline \( \varepsilon \dots 0 = \begin{pmatrix}
	6\\1\\-2
\end{pmatrix} \cdot \bigg( \begin{pmatrix}
	-1\\-2\\-5
\end{pmatrix} - \overrightarrow{x} \bigg) \)

\end{document}

